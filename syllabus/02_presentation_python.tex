Python est un langage de programmation inventé par Guido van Rossum en 1990. Son nom est une référence à la série télévisée Monty Python, dont van Rossum était fan.

Au fil des années, le langage a évolué. En 2000, on voit apparaître \texttt{Python 2.0}, et en 2008, \texttt{Python 3.0}.
Actuellement, deux version de Python coexistent : \texttt{Python 2.7}, et \texttt{Python 3.6}. Nous utiliserons dans ce cours la dernière version.

Python est un langage très utilisé. Sa syntaxe simple en fait un des langages les plus appréciés pour écrire des \textit{scripts}. Il est aussi utilisé dans le monde scientifique, dans des domaines tels que la science des données et l'intelligence artificielle.
