Les commentaires sont des indications que le programmeur donne à quiconque voudrait lire son code. Les parties de texte commentées sont totalement ignorées par l'ordinateur lors de l'exécution d'un programme et, ainsi, n'affecte pas le code. En \textsc{Python}, on commente une phrase en la précédant du symbole \# et on commente un texte en le précédant de """ et en le terminant avec """ .
\begin{python}
print "Hello"
#Cette phrase est ignoree
print "World" #Cette phrase est ignoree
"""Toute
cette
phrase
est
ignoree"""
\end{python}
\subsubsection{Exercice}
    Commenter le code de l'exercice précédent en précisant \textbf{l'auteur du code}, \textbf{la date de création}, \textbf{le lieu de création} et expliquer ce que fait l'instruction \texttt{print} pour quelqu'un n'ayant jamais fait de Python. 