L'instruction la plus simple est l'affichage de texte dans la console. Comme vous l'aurez deviné grâce à votre fonction test, l'affichage se fait grâce à l'instruction \texttt{print}.

\begin{python}
print("Le texte que je veux imprimer")
\end{python}

Il est préférable de ne pas mettre d'accents quand on imprime un texte dans la console, parce que toutes les configurations ne supportent pas leur affichage.

Pour effectuer un retour à la ligne, on peut terminer le texte par une suite de caractères spéciale : \lstinline{\n}.

Notez déjà qu'on peut imprimer autre chose que du texte brut : ce sont des variables, dont on parlera dans le chapitre suivant.
\subsubsection{Exercices}
    Modifier le code pour que l'ordinateur affiche "Bonjour maman"