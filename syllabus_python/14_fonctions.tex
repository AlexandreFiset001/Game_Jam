Une fonction est un outil informatique qui encapsule une portion de code (une séquence d'instructions) et qui effectue une tâche bien spécifique (calcul, avertissement, affichage, ...).

L'utilisation d'une fonction se nomme \textit{appel d'une fonction}. Chaque appel de fonction contient des \textbf{arguments} qui sont les informations avec laquelle la fonction va travailler.

 Ce concept n'est pas nouveau, vous connaissez déjà plein de fonctions! Par exemple, la fonction \texttt{print()} qui prend en argument un \texttt{str} et l'imprime dans la console. \\
 
 Une fonction peut renvoyer une valeur, résultat de son exécution. Cette valeur s'appelle \textbf{la valeur de retour}. Autre exemple, la fonction \texttt{input} qui prend en argument un \texttt{str} (qui est la question posée à l'utilisateur) et retourne un \texttt{str} (qui est la réponse de l'utilisateur).\\
 
 \subsection{Créer ses propres fonctions}
 Utiliser des fonctions toutes faites c'est bien. En créer soi-même, c'est mieux ! 
 En \textsc{Python}, la création de fonction se fait par le code \texttt{def nom\_de\_la\_fonction(argument1, argument2, ...):}. Le code indenté qui suit cette instruction est le \textit{corps} de la fonction et correspond aux instructions que celle-ci exécutera à chaque appel. Pour renvoyer une valeur, on utilise le mot clé \texttt{return} suivi de la valeur qu'on souhaite renvoyer. Enfin, pour appeler une fonction, on utilise le nom de la fonction suivi des arguments qu'on veut lui passer entre parenthèses.
 
\begin{python}[caption = Exemple 1 : La fonction addition]
def addition(arg1, arg2): #Definition de la fonction addition
	return arg1+arg2
print(addition(3,4)) #Affiche 7
\end{python}
\begin{python}[caption = Exemple 2 : La fonction reverse]
def reverse(my_string): #Definition de la fonction qui inverse un String
	toReturn = ""
	for i in my_string:
		toReturn = i + toReturn
	return toReturn
print(reverse("esarhp eugnol zessa enU")) #Affiche Une assez longue phrase 
\end{python}

 \subsection{Exercices :}
 \begin{enumerate}
 \item Créez les fonctions \texttt{soustraction(arg1, arg2)}, \texttt{multiplication(arg1, arg2)}, \texttt{division(arg1, arg2)}, et \texttt{power5(arg)} qui, comme leurs noms l'indique font respectivement les opérations de soustraction, multiplication, division, mise à la puissance de 5. 
 \item Créez la fonction \lstinline{string_length(my_string)} qui affiche la longueur du texte en argument.
 %\item Implémentez le jeu du touché-coulé. (En cours)
 \end{enumerate}