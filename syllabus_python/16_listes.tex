Une liste est une structure de données permettant de regrouper des données de manière à pouvoir y accéder librement.

\begin{python}[caption = Exemples de listes]
jeuxVideos = ['HeartStone', 'LoL', 'WoW', 'ClubPenguin']
chiffres = [0, 1, 2, 3, 4, 5, 6, 7, 8, 9]
\end{python}

Accéder à un élément d'une liste se fait de la manière que pour un \texttt{String} (on commence à \textbf{0}).
En effet, un \texttt{String} est en réalité une liste de caractères!

\begin{python}[caption = Accès à une liste]
chiffres = [0, 1, 2, 3, 4, 5, 6, 7, 8, 9]

chiffre0 = chiffres[0]      #chiffre0 vaut 0
chiffre3 = chiffres[3]      #chiffre3 vaut 3
chiffres3a6 = chiffres[3:7] #chiffres3a6 vaut [3 4 5 6]
\end{python}

On peut modifier une liste de plusieurs façons : changer, supprimer, ajouter un ou plusieurs éléments.

\begin{python}[caption = Modification de liste]
my_list = ['a', 'b', 'c', 'd', 'e']

#Modification
my_list[3] = 'f'    #['a', 'b', 'c', 'f', 'e']
#Suppression
del my_list[3]      #['a', 'b', 'c', 'e']
my_list.remove('e') #['a', 'b', 'c']
#Ajout
my_list.append('d') #['a', 'b', 'c', 'd']
my_list + ['e']     #['a', 'b', 'c', 'd', 'e']
\end{python}

Les listes sont également assorties de quelques opérations très pratiques et souvent employées.

\begin{python}[caption = Opérations sur les listes]
my_list = ['a', 'b', 'c', 'd', 'e']

#Longueur de la liste
longueur = len(my_list)  #longueur vaut 5
#Presence d un element
'd' in my_list #vaut True 
'f' in my_list #vaut False
#Parcourir une liste
for x in my_list : print (x) #imprime abcde
#Retourner une liste
my_list.reverse() #my_list vaut ['e', 'd', 'c', 'b', 'a']
#Mettre une liste dans l ordre croissant
my_list.sort() #my_list vaut ['a', 'b', 'c', 'd', 'e']
\end{python}

\subsubsection{Exercices}
    \begin{enumerate}
	\item
		Créez une liste de Strings contenant au moins 5 fruits et légumes, puis triez la dans l'ordre alphabétique.
	\item
		Ajoutez à votre liste les éléments suivants \textbf{s'ils ne sont pas déjà présents} :
		['banane', 'tomate', 'rutabaga', 'reine-claude', 'patate']
\item
		Éliminez de votre liste tous les éléments dont la première lettre commence par la lettre \textbf{p}.
\end{enumerate}