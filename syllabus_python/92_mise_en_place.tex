Un Raspberry Pi n'a pas de disque dur intégré, ni d'autre système de stockage fixe. A la place, il utilise une simple carte micro-SD. Une faible capacité suffit : 8Go ou 16Go sont amplement suffisants pour commencer. On stockera dessus le \textbf{système d'exploitation}. Si on prévoit d'utiliser le Raspberry Pi pour traiter des gros fichiers, on peut bien sûr connecter un disque dur externe, ou un autre type de stockage, via les ports USB de l'appareil.

Mais même en ayant installé un système d'exploitation, on ne pourra pas faire grand chose de cette petite carte verte... Il nous manque des périphériques élémentaires : un écran, un clavier, et une souris!
On connecte le clavier et la souris via les ports USB. Quant à l'écran, le Raspberry Pi peut y être connecté avec son port HDMI.
En pratique cependant, on accède souvent au Raspberry Pi à distance via un autre ordinateur, ainsi on n'a pas besoin de connecter un clavier ou un écran. Nous apprendrons à faire cela dans un prochain chapitre.

Enfin, il n'y a pas de bouton ON/OFF sur le Raspberry Pi. Pour l'allumer, il suffit de le brancher à un prise électrique via son cordon d'alimentation.